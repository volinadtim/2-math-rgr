\documentclass[class=article, crop=false]{standalone}
\usepackage[subpreambles=true]{standalone}

\usepackage{preamble}

\begin{document}

Исследуйте ряд Тейлора функции $f(x)$ в точке $x_0$. Изобразите графически несколько различных  частичных сумм ряда и график исходной функции. Проведите анализ полученных результатов.\\

Выполненная работа в Desmos доступна по 
\href{https://www.desmos.com/calculator/0sct6ckztm}{ссылке}

\subsubsection*{Ход работы}

\begin{enumerate}
    \item В заданной точке разложили  функцию в ряд Тейлора.
	
    $f(x) = (2 - e^{(x-2)})^2 = 4 - 4e^{(x-2)} + e^{2(x-2)}, x_0 = 2$

    Мы умеем раскладывать $g(x) = e^{x}$, если $x\to0 \Rightarrow$ мы умеем раскладывать $g(x) = e^{x-2}$ если $x \to 2$.\\

    Получим:

    $e^{x} =\sum_{n=0}^{\infty} \frac{x^n}{n!} \quad \text{при} \quad x \to 0$

    $e^{x-2} =        \sum_{n=0}^{\infty} \frac{(x-2)^n}{n!} \quad \text{при} \quad x \to 2$

    $e^{2(x-2)} =        \sum_{n=0}^{\infty} \frac{(x-2)^n2^{n}}{n!} \quad \text{при} \quad x \to 2$

    Получим:

    $f(x) = 4 + \sum_{n=0}^{\infty}(\frac{2^{n}}{n!} - \frac{4}{n!})(x-2)^{n} = 4 + \sum_{n=0}^{\infty}\frac{2^{n}-4}{n!}(x-2)^{n}$
    
    Ряд $e^{x-2}$ сходится на всей числовой оси $\Rightarrow$ и получившийся ряд тоже сходится на всей числовой оси.

    \item Нашли область сходимости полученного ряда к функции $f(x)$ - $(-\infty, \infty)$
    \item Построили графики частичных сумм ряда Тейлора (полиномов Тейлора) и график функции.
\end{enumerate}
\newpage

\end{document}