\documentclass[class=article, crop=false]{standalone}
\usepackage[subpreambles=true]{standalone}

\usepackage{preamble}

\begin{document}

\subsubsection*{Ход работы}

\begin{codelisting}
\inputminted{Python}{code/1.5.py}
\end{codelisting}

\subsubsection*{Вывод программы}

\begin{verbatim}
Эталонное значени интеграла - 1.6094379124341014
\end{verbatim}

\begin{tabular}{|l|c|c|}
     \hline
     Метод & Значение & Погрешность\\
     \hline\hline
     rectangular_method & 1.0897546897546897 & 0.5196832226794117 \\
     trapezoidal_method & 1.6833333333333333 & -0.0738954208992319\\
     simpson_method & 1.6222222222222213 & -0.012784309788119952\\
     weddle & 1.6094401258400164 & -2.2134059149969687e-06 \\
     \hline
\end{tabular}

\subsubsection*{Заключение}

Мы на практике поработали с численными методами вычислени определенных интегралов:

\begin{enumerate}
    \item Метод прямоугольников заключался в покрытии площади под графиков в прямоугольники, площадь которых мы хорошо знаем.
    \item Метод трапеций заключался в покрытии площади под графиком прямоугольными трапециями, крайние точки которых - точки графика.
    \item Метод парабол заключался в апроксимации равномерных частей графика функции параболами, определенный интеграл которых мы легко находим из формулы Ньютона-Лейбница. 
    \item Метод Уэддля в своей сути - метод парабол, но апроксимация идет по 7 степеням многочлена
\end{enumerate}

Как мы можем вдиеть из вывода программы, методы располагаются от менее точного, к более точному
\newpage

\end{document}