\documentclass[class=article, crop=false]{standalone}
\usepackage[subpreambles=true]{standalone}

\usepackage{preamble}

\begin{document}

\subsubsection*{Ход работы}

\begin{codelisting}
\begin{minted}{Python}
import numpy
from scipy import integrate


def f(x):
    return 1.0 / (2.0 + x)


def rectangular_method(a, b):
    n = b - a  # количество отрезков (при h = 1)
    h = 1
    I = 0  # интегральная сумма
    x = a + h / 2
    for i in range(n):
        x += h  # левая граница
        I += f(x)  # ((b - a) / n) * f(x(i-1) + h / 2)

    return I

def trapezoidal_method(a, b):
    h = 1
    n = b - a // h  # количество отрезков (при h = 1)
    I = 0
    for i in range(n):
        x1 = a + h * i
        x2 = x1 + h
        I += (f(x1) + f(x2)) / 2 * (x2 - x1)
    return I

def find_para(xi1, xi2, xi3):
    a = 0
    b = 0
    c = 0
    x = numpy.array([xi1, xi2, xi3])
    y = numpy.array([f(xi1), f(xi2), f(xi3)])
    z = numpy.polyfit(x, y, 2)
    return z[0], z[1], z[2]


def integrte_para(a, b, c, start, end):
    return ((a * end ** 3) / 3 + (b * end ** 2) / 2 + c * end) - \
        ((a * start ** 3) / 3 + (b * start ** 2) / 2 + c * start)


def simpson_method(a, b):
    I = 0
    h = 1
    n = (b - a) // (2 * h)
    for i in range(n):
        xi1 = a + 2 * h * i
        xi2 = xi1 + h
        xi3 = xi1 + 2 * h
        a_p, b_p, c_p = find_para(xi1, xi2, xi3)
        I += integrte_para(a_p, b_p, c_p, xi1, xi3)
    return I


def weddle(f, a, b, n):
    h = (b - a) / n
    s = 0.0
    x1 = a
    x2 = a + h
    while (x2 <= b):
        dx = (x2 - x1) / 6
        s = s + (h / 20) * (f(x1) + 5 * f(x1 + dx) + f(x1 + 2 * dx) + \
        6 * f(x1 + 3 * dx) + f(x1 + 4 * dx) + 5 * f(x1 + 5 * dx) + f(x1 + 6 * dx))
        x1 = x2
        x2 = x1 + h
    return s

a = -1
b = 3
result = rectangular_method(a, b)
predicted = integrate.quad(f, a, b)[0]

print("{} rectangular_method || fault {}".format(result, predicted - result))

result = trapezoidal_method(a, b)

print("{} trapezoidal_method || fault {}".format(result, predicted - result))

result = integrate.quad(f, a, b)

print("{} library".format(result))

result = simpson_method(a, b)

print("{} simpson_method || fault {}".format(result, predicted - result))

result = weddle(f, a, b, 4)

print("{} weddle || fault {}".format(result, predicted - result))
\end{minted}
\end{codelisting}

\subsubsection*{Вывод программы}

\begin{verbatim}
Эталонное значени интеграла - 1.6094379124341014
\end{verbatim}

\begin{tabular}{l||c||c}
     \hline
     Метод & Значение & Погрешность\\
     \hline\hline
     rectangular_method & 1.0897546897546897 & 0.5196832226794117 \\
     trapezoidal_method & 1.6833333333333333 & -0.0738954208992319\\
     simpson_method & 1.6222222222222213 & -0.012784309788119952\\
     weddle & 1.6094401258400164 & -2.2134059149969687e-06
\end{tabular}

\subsubsection*{Заключение}

Мы на практике поработали с численными методами вычислени определенных интегралов:

\begin{enumerate}
    \item Метод прямоугольников заключался в покрытии площади под графиков в прямоугольники, площадь которых мы хорошо знаем.
    \item Метод трапеций заключался в покрытии площади под графиком прямоугольными трапециями, крайние точки которых - точки графика.
    \item Метод парабол заключался в апроксимации равномерных частей графика функции параболами, определенный интеграл которых мы легко находим из формулы Ньютона-Лейбница. 
    \item Метод Уэддля в своей сути - метод парабол, но апроксимация идет по 7 степеням многочлена
\end{enumerate}

Как мы можем вдиеть из вывода программы, методы располагаются от менее точного, к более точному
\newpage

\end{document}