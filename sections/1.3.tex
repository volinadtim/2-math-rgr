\documentclass[class=article, crop=false]{standalone}
\usepackage[subpreambles=true]{standalone}

\usepackage{preamble}

\begin{document}



% \begin{wrapfigure}{i}{\textwidth}
\begin{center}
    \begin{tikzpicture}
    \begin{axis}[
        xlabel=$x$,
        ylabel={$y$},
        domain=-2:2,
        samples=200,
        axis lines=middle,
        ymin=-1,
        ymax=3
    ]
    \addplot [thick,color=Red] {rad(atan(x)/x)};
    \addplot [thick,color=Blue] gnuplot {x**2 * atan(x)};
    \addplot [thick,color={rgb,255:red,0;green,128;blue,0}] {rad(atan(x))/(x^3)};
    \node [above] at (axis cs:1.7,0.65) {\rotatebox{-10}{$\alpha = 1$}};
    \node [left] at (axis cs:-0.7,2.0) {\rotatebox{70}{$\alpha = 3$}};
    \node [left] at (axis cs:0,-0.3) {\rotatebox{20}{$\alpha = -2$}};
    \end{axis}
    \end{tikzpicture}
    
    \caption{ $
    f\left(x\right)=\frac{\arctg\left(x\right)}{x^{\alpha}}$}
    \label{fig:task3}
\end{center}
% \end{wrapfigure}

Выполненная работа в Desmos доступна по 
\href{https://www.desmos.com/calculator/jjtlvpwjjv?lang=ru}{ссылке}

\subsubsection*{Ход работы}

\begin{enumerate}
    \item Особой точкой подынтегральной функции является точка $x_0 = 0$, так как в ней происходит деление на ноль.
    \item На промежутке от 0 до 1 подынтегральная функция всегда принимает положительные значения, при любых $\alpha$.
    \item При $\alpha = 0$ интеграл легко вычисляется. Рассмотрим его:
    
    \begin{aligned}
    &\int\limits_{0}^{1}\frac{\arctg{x}}{x^0}dx = \int\limits_{0}^{1}\arctg{x}dx = x\arctg x\bigg|\limits_{0}^{1} - \frac{1}{2}\left(\ln\left(1 + x^2\right) \bigg|\limits_{0}^{1} \right) = \hfill \\
    &= 1\cdot \arctg{1} - 0\cdot\arctg{0} - \frac{1}{2} \left(\ln{2} - \ln{0}\right) = \frac{\pi -2\ln{2}}{4}
    \end{align}
    
    Отсюда можно сделать вывод, что интеграл сходится при $\alpha = 0$ и его значения $\approx 0.4388$.

    \item Сформулируем признаки сравнения:
    
    \begin{aligned}
        \text{Пусть функции } f(x) \text{ и } g(x) \text{ определены на промежутке } \\
        (A,B) \text{ и удовлетворяют неравенству } f(x)<=\abs(g(x))
        
    \end{aligned}

    \item Сделаем оценку сверху:
    $$
    \frac{\arctg{x}}{x^{\alpha}} < \frac{\frac{\pi}{2}}{x^{\alpha}}
    $$
    Отсюда видно, что при $\alpha<1$ интеграл от правой функции будет сходится, а значит интеграл от меньшей функции тоже сходится.

    \item Оценка снизу, очевидно, $y = 0$.

\end{enumerate}

\subsubsection*{Заключение}

Тут какие-то словечки...

\newpage

\end{document}