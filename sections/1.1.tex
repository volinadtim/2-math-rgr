\documentclass[class=article, crop=false]{standalone}
\usepackage[subpreambles=true]{standalone}

\usepackage{preamble}

\begin{document}

% \begin{wrapfigure}{i}{\textwidth}
\begin{center}
    \begin{tikzpicture}
    \begin{axis}[
        xlabel=$x$,
        ylabel={$y$},
        domain=-3:3,
        samples=200,
        axis lines=middle,
        ymin=-1,
        ymax=3
    ]
    \addplot [thick,color=Red] {1/(cos(deg(x)))^2};
    \addplot [dashed] coordinates {(pi/4,0) (pi/4,3)};
    \addplot [dashed] coordinates {(-pi/4,0) (-pi/4,3)};
    \end{axis}
    \end{tikzpicture}
    
    \caption{График функции $f(x) = \frac{1}{\cos^2{x}}$}
    \label{fig:task1}
\end{center}
% \end{wrapfigure}

\subsection{Интегральная сумма}

Выполненная работа в Desmos доступна по 
\href{https://www.desmos.com/calculator/4mxkiocqhk?lang=ru}{ссылке}

\subsubsection*{Ход работы}

\begin{enumerate}
    \item Построили в Desmos интегральную сумму и график функции.
    Сделали все необходимые ползунки для изменения количества точек в разбиении
    и смещения точек внутри элементарных отрезков.
    \item Исследовали ступенчатую фигуру при количестве ступеней 5, 10 и 100. В каждом положении рассматривали фигуру в левом крайнем, правом крайнем и промежуточном положениях точек внутри элементарных отрезков.

    $f(x) = \frac{1}{cos^2{x}}$

    $\int_{a}^{b}f\left(x\right)dx = 2$

    $l = \frac{b-a}{m}$, m - количество отрезков, a - левая граница, b - правая граница, l - величина одного отрезка, h = оснащение

    Интегральная сумма для 5 отрезков:

    $\sum_{p=0}^{4}f\left(a+\left(\frac{b-a}{5}\right)\left(h+p\right)\right)\cdot\frac{b-a}{5} = 1.98352670512$

    Интегральная сумма для 10 отрезков:

    $\sum_{p=0}^{9}f\left(a+\left(\frac{b-a}{10}\right)\left(h+p\right)\right)\cdot\frac{b-a}{10} = 1.99576571033$

    Интегральная сумма для 100 отрезков:
    
    $\sum_{p=0}^{99}f\left(a+\left(\frac{b-a}{100}\right)\left(h+p\right)\right)\cdot\frac{b-a}{100} = 1.99995723608$
\end{enumerate}

\subsubsection*{Заключение}

\quad \quad На графике легко видеть, что при увеличении количества точек в разбиении, ступенчатая фигура наиболее точно повторяет график, а значение интегральных сумм становится ближе к реальному значению интеграла.

\quad Так как график симметричный, то крайне левое и крайне правое положение точек внутри элементарных отрезков давали зеркальные фигуры, поэтому выбор соответстующего оснащения дает одинаковый результат.


\subsection{Последовательность интегральных сумм}

Выполненная работа в Desmos доступна по 
\href{https://www.desmos.com/calculator/3zjcyijmmm?lang=ru}{ссылке}

\subsubsection*{Ход работы}

\begin{enumerate}
    \item Задали в Desmos формулу для интегральной суммы и отоюразили на графике множество её значений при разном $n$ - количестве точек в разбиении.
    \item Рассмотрели значения интегральной суммы при росте $n$ и разном положении точек внутри одного отрезка. Это легко проделать, двигая ползунки в Desmos.
    \item Вычислили интеграл от данной функции аналитически:
    
    \begin{aligned}
    & \int\limits_{-\frac{\pi}{4}}^{\frac{\pi}{4}}\frac{1}{\cos^2{x}}dx = \int\limits_{-\frac{\pi}{4}}^{\frac{\pi}{4}}(1+\tg^2{x})dx = x\bigg|\limits_{-\frac{\pi}{4}}^{\frac{\pi}{4}} + \int\limits_{-\frac{\pi}{4}}^{\frac{\pi}{4}}(\tg^2{x})dx = \left|
    \begin{aligned}
    t = \tg x \\ x = \arctg t \\ dx = \frac{1}{1+t^2}dt \\ t_1 = 1, t_2 = -1 
    \end{aligned}
    \right| = \\ 
    &= \frac{\pi}{2} + \int\limits_{-1}^{1}\frac{t^2}{1+t^2}dt = \frac{\pi}{2} + t\bigg|\limits_{-1}^{1} - \int\limits_{-1}^{1}\frac{1}{1+t^2}dt = \frac{\pi}{2} + 2 - \arctg x\bigg|\limits_{-1}^{1} = \\
    &= \frac{\pi}{2} + 2 - (\frac{\pi}{4} - (-\frac{\pi}{4})) = \frac{\pi}{2} + 2 - \frac{\pi}{2} = 2
    \end{aligned}

    \item Изобразили значение интеграла на графике.
\end{enumerate}

\subsubsection*{Заключение}

\quad На графике можно легко увидеть, что чем больше точек в разбиении мы берём, тем точнее интегральная сумма повторяет значение интеграла от данной функции, вычисленного аналитически. Можно заметить, что вне зависимости от выбора положения точки внутри элементарного отрезка, при больших $n$ интегральная сумма настолько приближается к значению интеграла, что без увеличения масштаба различия не видны. При значении $n=100$ интегральная сумма отличается от интеграла не более чем на $0.0002$ при выборе точки в крайне левом положении. При малых значениях $n$ можно заметить, как именно выбор положения точки внутри элементарного отрезка влияет на интегральную сумму. Так как наша функция чётная, то крайне левое и крайне правое положения совпадают. При малых $n$ и выборе крайних положений точки внутри элементарного отрезка интегральная сумма становится заметно больше значения интеграла. При больших $n$ и большом масштабе видно, что интегральная сумма будет превышать значения интеграла, но на малые величины. Также можно подобрать такое оснащение, что даже при малых $n$ интегральная сумма будет приблизительно равна интегралу, хотя на практике чаще всего для удобства берутся крайние или среднее значения.

\newpage

\end{document}