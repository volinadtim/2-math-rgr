%\documentclass{beamer}
% \documentclass[paper=smartboard]{powerdot}
\documentclass[fleqn]{article}

\usepackage{siunitx}
\usepackage[russian,english]{babel}

\usepackage[T2A]{fontenc}
\usepackage[utf8]{inputenc}

\usepackage{setspace}

\usepackage{amsmath}

\usepackage{hyperref}
\usepackage{hyphenat}

\usepackage{breqn}

\usepackage{graphicx} % Required for inserting images

\usepackage{pgfplots}

\usepackage{xcolor}
\definecolor{Red}{HTML}{CA4F4B}
\definecolor{Blue}{HTML}{2D70B3}
\definecolor{Orange}{HTML}{FA7E19}

\usepackage{wrapfig}

\hypersetup{
  pdfborder = {0 0 0},
  colorlinks=true,
  linkcolor=black
}


\makeatletter
\renewcommand{\l@section}{\@dottedtocline{1}{0.0em}{0.0em}}
\renewcommand{\l@subsection}{\@dottedtocline{2}{1.4em}{5.5em}}
\renewcommand{\l@subsubsection}{\@dottedtocline{3}{7.4em}{4.5em}}
\makeatother


\renewcommand{\thesection}{}

\renewcommand{\thesubsection}{Задание \arabic{subsection}.}

\addtocontents{toc}{\protect\renewcommand{\protect\thesubsection}{\arabic{subsection}}}


\begin{document}

\begin{titlepage}
    \centering
    \vspace*{0.5 cm}
    \textsc{\LARGE Математический анализ}\\[1.0 cm]
    \textsc{\Large Первый модуль}\\[0.5 cm]
    \textsc{\large 2022-2023 учебный год}\\[1.5 cm]
    \rule{\linewidth}{0.1 mm} \\[0.4 cm]
    { \huge \bfseries Интеграл функции от одной переменной. Вариант 17}\\[0.2 cm]
    \rule{\linewidth}{0.1 mm} \\[2 cm]
    \begin{minipage}{0.6\textwidth}
        \begin{flushleft} \large
            \emph{Автор:}\\
            Данилов Тимофей 1.1
        \end{flushleft}
    \end{minipage}~
    \begin{minipage}{0.4\textwidth}
        \begin{flushright} \large
            \emph{Дата:} \\
            3 июня 2023
        \end{flushright}
    \end{minipage}\\[2 cm]
    {\large Университет ИТМО}\\[2 cm]
    \vfill
\end{titlepage}

\renewcommand{\contentsname}{Содержимое}
\tableofcontents
\newpage


\section{Задания}

\subsection{Значение функции}

Вычислить приближенное значение функции $f(x) = \ln(x)$ в точке $x_0 = \frac{9}{8}$ c точностью 0,0001.

\subsection{Вычислить интеграл}

Разлагая подынтегральную функцию в степенной ряд вычислить приблеженно интеграл $\int\limits_{0}^{1/3} \frac{x}{\sqrt{1+x^3}}dx$ с точностью 0,0001.

\subsection{Степенной ряд}

Найти в виде степенного ряда решение дифференциального уравнения, удовлетворяющего заданным начальным условиям. Ограничиться четырьмя членами ряда. 

\begin{center}
    $y'' = y'\sin{y};   y(2) = \frac{\pi}{2}, y'(2) = 1$
\end{center}

\renewcommand{\thesubsection}{\arabic{subsection}.}


\section{Задание 1. Значение функции $f(x) = \ln(x)$}

Выполненная работа в Desmos доступна по 
\href{https://www.desmos.com/calculator/bubpcmb9ie}{ссылке}

\subsection*{Ход работы}

\[
\ln{(1+x)} = x - \frac{x^2}{2} + \frac{x^3}{3} - \frac{x^4}{4} + \frac{x^5}{5} - \frac{x^6}{6} ... \text{если} |x| \leq 1
\]

Итак, $x = \frac{1}{8}$, в этой точке логарифм сходится \Rightarrow можно исопльзовать разложение в ряд.

Далее $\left(\frac{1}{8}\right)^5$ уже дает нам необходимую точность.

\[
\ln{\left(\frac{9}{8}\right)} \approx \frac{1}{8} - \frac{\left(\frac{1}{8}\right)^2}{2} + \frac{\left(\frac{1}{8}\right)^3}{3} - \frac{\left(\frac{1}{8}\right)^4}{4} = 0,1177
\]

\section{Задание 2. Вычислить интеграл $\int\limits_{0}^{1/3} \frac{x}{\sqrt{1+x^3}}dx$}

График в Desmos доступен по
\href{https://www.desmos.com/calculator/9n57v4dodl}{ссылке}


\begin{multiline*}
\int\limits_{0}^{\frac{1}{3}} \frac{x}{\sqrt{1+x^3}}dx = \int\limits_{0}^{\frac{1}{3}} x \cdot (1+x^3)^{-0.5}dx &&\\
(1 + x)^{\alpha} =  \sum\limits_{0}^{\infty} C_{n}^{\alpha} x^{n} &&\\
(1 + x^3)^{-\frac{1}{2}} =  \sum\limits_{0}^{\infty} C_{n}^{-\frac{1}{2}} x^{3n} &&\\
\int\limits_{0}^{\frac{1}{3}} x \sum\limits_{0}^{\infty} C_{n}^{-\frac{1}{2}} x^{3n}dx = \sum\limits_{0}^{\infty}\int\limits_{0}^{\frac{1}{3}}  C_{n}^{-\frac{1}{2}} x^{3n+1}dx = \sum\limits_{0}^{\infty} C_{n}^{-\frac{1}{2}} \int\limits_{0}^{\frac{1}{3}}   x^{3n+1}dx = \sum\limits_{0}^{\infty} C_{n}^{-\frac{1}{2}} \left(  \frac{x^{3n+2}}{3n+2} \Big|_{0}^{\frac{1}{3}} \right) = \sum\limits_{0}^{\infty} C_{n}^{-\frac{1}{2}} \frac{\left(\frac{1}{3}\right)^{3n+2}}{3n+2} \approx 0,0551
\end{multiline*}

\section{Задание 3. Степенной ряд}

Выполненная работа в Desmos доступна по 
\href{https://www.desmos.com/calculator/0imxfkpgnf}{ссылке}

\subsection*{Ход работы}

$y(1) = 2, \quad y'(1) = 1, \quad y'' = y\ln{y'}$

$\quad\sqsupset y = y(1) + \frac{y'(1)(x-1)}{1!} + \frac{y''(1)(x-1)^2}{2!} + \frac{y'''(1)(x-1)^3}{3!}$ - Ряд Тейлора для данной функции в точке $x_0 = 1$, ограничивающийся только 4 первыми слагаемыми, итак значения 0 и 1 производных мы знаем, рассамотрим производные далее:
\[
    y''(2) = y\ln{y'}
\]
\[
    y''' = (y'')' = y'\ln{y'} + \frac{y * y''}{y'}
\]
Итого получим:
$y = 2 + x - 1 = x + 1$

\end{document}