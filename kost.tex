%\documentclass{beamer}
% \documentclass[paper=smartboard]{powerdot}
\documentclass{article}

\usepackage{siunitx}
\usepackage[russian,english]{babel}

\usepackage[T2A]{fontenc}
\usepackage[utf8]{inputenc}

\usepackage{setspace}

\usepackage{amsmath}

\usepackage{hyperref}
\usepackage{hyphenat}

\usepackage{breqn}

\usepackage{graphicx} % Required for inserting images

\usepackage{pgfplots}

\usepackage{xcolor}
\definecolor{Red}{HTML}{CA4F4B}
\definecolor{Blue}{HTML}{2D70B3}
\definecolor{Orange}{HTML}{FA7E19}

\usepackage{wrapfig}

\hypersetup{
  pdfborder = {0 0 0},
  colorlinks=true,
  linkcolor=black
}


\makeatletter
\renewcommand{\l@section}{\@dottedtocline{1}{0.0em}{0.0em}}
\renewcommand{\l@subsection}{\@dottedtocline{2}{1.4em}{5.5em}}
\renewcommand{\l@subsubsection}{\@dottedtocline{3}{7.4em}{4.5em}}
\makeatother


\renewcommand{\thesection}{}

\renewcommand{\thesubsection}{Задание \arabic{subsection}.}

\addtocontents{toc}{\protect\renewcommand{\protect\thesubsection}{\arabic{subsection}}}


\begin{document}

\begin{titlepage}
    \centering
    \vspace*{0.5 cm}
    \textsc{\LARGE Математический анализ}\\[1.0 cm]
    \textsc{\Large Первый модуль}\\[0.5 cm]
    \textsc{\large 2022-2023 учебный год}\\[1.5 cm]
    \rule{\linewidth}{0.1 mm} \\[0.4 cm]
    { \huge \bfseries Интеграл функции от одной переменной. Вариант 16}\\[0.2 cm]
    \rule{\linewidth}{0.1 mm} \\[2 cm]
    \begin{minipage}{0.6\textwidth}
        \begin{flushleft} \large
            \emph{Автор:}\\
            Бушмелев Константин 1.1
        \end{flushleft}
    \end{minipage}~
    \begin{minipage}{0.4\textwidth}
        \begin{flushright} \large
            \emph{Дата:} \\
            3 июня 2023
        \end{flushright}
    \end{minipage}\\[2 cm]
    {\large Университет ИТМО}\\[2 cm]
    \vfill
\end{titlepage}

\renewcommand{\contentsname}{Содержимое}
\tableofcontents
\newpage


\section{Задания}

\subsection{Значение функции}

Вычислить приближенное значение функции $f(x) = 3^x$ c точностью 0,0001.

\subsection{Вычислить интеграл}

Разлагая подынтегральную функцию в степенной ряд вычислить приблеженно интеграл $\int\limits_{0}^{\frac{1}{3}} x^3 \sqrt[\leftroot{4} 4]{1 + x^2}dx$ с точностью 0,0001.

\subsection{Степенной ряд}

Найти в виде степенного ряда решение дифференциального уравнения. Получить интегральную траекторию, использую заданные начальные условия (Ограничиться четырьмя первыми ненулевыми членами ряда). Изобразить на графике полученное общее решение и решение задачи Коши.


\renewcommand{\thesubsection}{\arabic{subsection}.}


\section{Задание 1. Значение функции $f(x) = 3^x$}

\[
f(x) = 3^x
\]

\[
a^x = e^{x\ln{a}}
\]

\[
e^x = \sum^\infty_{n=0}\frac{x^n}{n!}, \quad a_n = \frac{x^n}{n!}
\]

\[
e^{x\ln{a}} = e^{\frac{2\ln{3}}{3}} = 1 + \frac{\frac{2}{3}\ln{3}}{1!}  + \frac{(\frac{2}{3}\ln{3})^2}{2!} + \frac{(\frac{2}{3}\ln{3})^3}{3!} + \frac{(\frac{2}{3}\ln{4})^2}{4!} + \frac{(\frac{2}{3}\ln{3})^5}{5!} + \frac{(\frac{2}{3}\ln{3})^6}{6!} + r_n
\]

Докажем, что остаток меньше $\epsilon$

\[
|r_n| = |a_{n+1}| + |a_{n+2}| + \dots
\]

\[
a_{n+1} = a_n\frac{x}{n+1}
\]

\[
a_{n+i} = a_n\frac{x^i}{n+i}
\]

\[
|r_n| = |a_n|\frac{|x|}{n+1} + \dots + |a_n|\frac{|x^i|}{n+1}| + \dots = \frac{\frac{|x|}{n+1}|a_n|}{1-\frac{|x|}{n+1}} = \frac{|x||a^n|}{(n+1) - |x|} = \frac{\frac{(\frac{2\ln{3}}{3})^6}{6!}\frac{2}{3}}{7-\frac{2}{3}} \approx 0,000022 \dots < 0,0001
\]

\[
|r_n| < 0,0001 
\]

\[
f(x) = 1 + \frac{\frac{2}{3}\ln{3}}{1!}  + \frac{(\frac{2}{3}\ln{3})^2}{2!} + \frac{(\frac{2}{3}\ln{3})^3}{3!} + \frac{(\frac{2}{3}\ln{3})^2}{4!} + \frac{(\frac{2}{3}\ln{3})^5}{5!} + \frac{(\frac{2}{3}\ln{3})^6}{6!} \approx 2.0801 \dots
\]

% \subsubsection*{Заключение}

% Тут какие-то словечки...


\section{Задание 2. Вычислить интеграл $\int\limits_{0}^{\frac{1}{3}} x^3 \sqrt[\leftroot{4} 4]{1 + x^2}dx$}

\[ 
\int\limits_{0}^{\frac{1}{3}} x^3  \sqrt[\leftroot{4} 4]{1 + x^2}dx 
\]

\[ 
(1 + x)^{\alpha} = \sum\limits_{0}^{\infty} C_{n}^{\alpha} x^{n} 
\]

\[ 
(1 + x^2)^{\frac{1}{4}} =  \sum\limits_{0}^{\infty} C_{n}^{\frac{1}{4}} x^{2n} 
\]

\[ 
\begin{split}
\int\limits_{0}^{\frac{1}{3}} x^3 \sum\limits_{0}^{\infty} C_{n}^{\frac{1}{4}} x^{2n}dx = 
\sum\limits_{0}^{\infty}\int\limits_{0}^{\frac{1}{3}}  C_{n}^{\frac{1}{4}} x^{2n+3}dx = 
\sum\limits_{0}^{\infty} C_{n}^{\frac{1}{4}} \int\limits_{0}^{\frac{1}{3}}   x^{2n+3}dx &= \\
\sum\limits_{0}^{\infty} C_{n}^{\frac{1}{4}} \left(  \frac{x^{2n+4}}{2n+4} \Big|_{0}^{\frac{1}{3}} \right) &= \sum\limits_{0}^{\infty} C_{n}^{\frac{1}{4}} \frac{\left(\frac{1}{3}\right)^{2n+4}}{2n+4} \approx 0,0031 
\end{split}
\]


\section{Задание 3. Степенной ряд}

\[ 
y(2) = 2, \quad y'(2) = 1, \quad y'' - (y')^2 = x
\]

$\quad\sqsupset y = y(2) + \frac{y'(2)(x-2)}{1!} + \frac{y''(2)(x-2)^2}{2!} + \frac{y'''(2)(x-2)^3}{3!}$ - Ряд Тейлора для данной функции в точке $x_0 = 2$, ограничивающийся только 4 первыми слагаемыми, итак значения 0 и 1 производных мы знаем, рассамотрим производные далее:
\[
    y''(2) = y'(2)^2 + x
\]
\[
    y''' = (y'')' = (y'(2)^2 + x)' = 2y'(2) + 1
\]  

Итого получим:

\[
y = 2 + x - 2 + \frac{1}{2}(x^3-3x^2+4) + \frac{1}{2}(x^3 - 2x^2 + 4x - 8) = \frac{x^3}{2}-\frac{5x^2}{2}+3x-2
\]

\end{document}
