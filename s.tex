%\documentclass{beamer}
% \documentclass[paper=smartboard]{powerdot}
\documentclass[fleqn]{article}

\usepackage{siunitx}
\usepackage[russian,english]{babel}

\usepackage[T2A]{fontenc}
\usepackage[utf8]{inputenc}

\usepackage{setspace}

\usepackage{amsmath}

\usepackage{hyperref}
\usepackage{hyphenat}

\usepackage{breqn}

\usepackage{graphicx} % Required for inserting images

\usepackage{pgfplots}

\usepackage{xcolor}
\definecolor{Red}{HTML}{CA4F4B}
\definecolor{Blue}{HTML}{2D70B3}
\definecolor{Orange}{HTML}{FA7E19}

\usepackage{wrapfig}

\hypersetup{
  pdfborder = {0 0 0},
  colorlinks=true,
  linkcolor=black
}


\makeatletter
\renewcommand{\l@section}{\@dottedtocline{1}{0.0em}{0.0em}}
\renewcommand{\l@subsection}{\@dottedtocline{2}{1.4em}{5.5em}}
\renewcommand{\l@subsubsection}{\@dottedtocline{3}{7.4em}{4.5em}}
\makeatother


\renewcommand{\thesection}{}

\renewcommand{\thesubsection}{Задание \arabic{subsection}.}

\addtocontents{toc}{\protect\renewcommand{\protect\thesubsection}{\arabic{subsection}}}


\begin{document}

\begin{titlepage}
    \centering
    \vspace*{0.5 cm}
    \textsc{\LARGE Математический анализ}\\[1.0 cm]
    \textsc{\Large Первый модуль}\\[0.5 cm]
    \textsc{\large 2022-2023 учебный год}\\[1.5 cm]
    \rule{\linewidth}{0.1 mm} \\[0.4 cm]
    { \huge \bfseries Интеграл функции от одной переменной. Вариант 24}\\[0.2 cm]
    \rule{\linewidth}{0.1 mm} \\[2 cm]
    \begin{minipage}{0.6\textwidth}
        \begin{flushleft} \large
            \emph{Автор:}\\
            Нигаматуллин Степан 1.1
        \end{flushleft}
    \end{minipage}~
    \begin{minipage}{0.4\textwidth}
        \begin{flushright} \large
            \emph{Дата:} \\
            3 июня 2023
        \end{flushright}
    \end{minipage}\\[2 cm]
    {\large Университет ИТМО}\\[2 cm]
    \vfill
\end{titlepage}

\renewcommand{\contentsname}{Содержимое}
\tableofcontents
\newpage


\section{Задания}

\subsection{Значение функции}

Вычислить приближенное значение функции $f(x) = \cos(x)$ в точке $x_0 = \frac{8}{9}$ c точностью 0,0001.

\subsection{Вычислить интеграл}

Разлагая подынтегральную функцию в степенной ряд вычислить приблеженно интеграл $\int\limits_{0}^{1} \frac{xdx}{\sqrt{16+x^4}}$ с точностью 0,0001.

\subsection{Степенной ряд}

Найти в виде степенного ряда решение дифференциального уравнения, удовлетворяющего заданным начальным условиям. Ограничиться четырьмя членами ряда. 
\begin{center}
    $y'' = y'\sin{y};   y(2) = \frac{\pi}{2}, y'(2) = 1$
\end{center}

\renewcommand{\thesubsection}{\arabic{subsection}.}


\section{Задание 1. Значение функции $f(x) = \cos(x)$}

Выполненная работа в Desmos доступна по 
\href{https://www.desmos.com/calculator/fvkf5exurh}{ссылке}

\subsection*{Ход работы}

\[
f\left(x\right)\ =\sum_{n=0}^{k}\frac{\left(-1\right)^{n}x^{2n}}{\left(2n\right)!}
\]

Заметим, что данный ряд сходится на всей числовой оси, следовательно, в частности, его можно применять если $x \Rightarrow \frac{8}{9}$, Заметим так же, что точность слагаемых ряда с 8 и 10 степенью составляет 4 и 6 знаков соответственно, следовательно, при нашей точности, брать слагаемые выше 8 степени не имеет смысла.

Приближенное значение: $cos(\frac{8}{9}) \approx \sum_{n=1}^{6}\frac{\left(-1\right)^{n+1}}{n}x^{n} = 0.6302$

% \subsubsection*{Заключение}

% Тут какие-то словечки...


\section{Задание 2. Вычислить интеграл $\int\limits_{0}^{1} \frac{xdx}{\sqrt{16+x^4}}$}

Выполненная работа в Desmos доступна по 
\href{https://www.desmos.com/calculator/pv1rt3ixlw}{ссылке}

\subsection*{Ход работы}

Разложим для начала функцию

\[
g(x) = \frac{1}{\sqrt{16+x}}
\]

\[
g(0) = \frac{1}{4}
\]

\[
g'(x) = - \frac{1}{16+x} \cdot \frac{1}{2\sqrt{16+x}}
\]

\[
g'(0) = - \frac{1}{16} \cdot \frac{1}{8} = - \frac{1}{128}
\]

\[
g''(x) = \left( -\frac{1}{2 \left( 16+x \right)^{\frac{2}{3}}} \right)' = -\frac{1}{2} \left( (16+x)^{-\frac{3}{2}}\right)' = \frac{1}{2} \cdot \frac{3}{2} \cdot (16+x)^{-\frac{5}{2}}
\]

\[
g''(x) = \frac{3}{4} \cdot 16^{-\frac{5}{2}} = \frac{3}{4} \cdot 16^{-\frac{5}{2}} = \frac{3}{4 \cdot 4^5} = \frac{3}{4^6} = \frac{3}{4096}
\]


Итого:

\[
    g(x) = \frac{1}{\sqrt{16+x}} = \frac{1}{4} - \frac{x}{128} + \frac{3}{8192} x^2 + O(x^3)
\]

Как можно заметить, $\frac{3}{8192}$ делает уже погрешность меньше чем нам необходимо, получим

\[
    g(x^4) = \frac{1}{4} - \frac{x^4}{128} + \frac{3}{8192}x^8 + O(x^12)
    &&\\ x \cdot g(x^4) = f(x) = \frac{1}{4}x - \frac{1}{128}x^5 + \frac{3}{8192}x^9 + O(x^13)
\]

Получим

\[
    \int\limits_{0}^{1}f(x)dx = \frac{1}{8}x^2 - \frac{1}{128 \cdot 6}x^6 + \frac{3}{81920}x^{10} \Big|_{0}^{1} = \frac{1}{8} - \frac{1}{128 \cdot 6} + \frac{3}{81920} = 0,12373 ...
\]

\[
\int\limits_{0}^{1} \frac{x}{\sqrt{16+x^4}} = 0,12373
\]


\section{Задание 3. Степенной ряд}

Выполненная работа в Desmos доступна по 
\href{https://www.desmos.com/calculator/0imxfkpgnf}{ссылке}

\subsection*{Ход работы}

$\quad\sqsupset y = y(2) + \frac{y'(2)(x-2)}{1!} + \frac{y''(2)(x-2)^2}{2!} + \frac{y'''(2)(x-2)^3}{3!}$ - Ряд Тейлора для данной функции в точке $x_0 = 2$, ограничивающийся только 4 первыми слагаемыми, итак значения 0 и 1 производных мы знаем, рассамотрим производные далее:

\[
    y''(2) = y'(2)\sin{y(2)} = 1
\]

\[
    y''' = (y'')' = (y'\sin{y})' = y''\sin{y} + \cos{y}y' \Rightarrow y'''(2) = 1
\]

Итого получим: $y = \frac{\pi}{2} - 2 + x + \frac{1}{2}(x-2)^2 + \frac{1}{6}(x-2)^3$

\end{document}