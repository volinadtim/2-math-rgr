\documentclass{article}

\usepackage[subpreambles=true]{standalone}
\usepackage{import}

\usepackage{siunitx}
\usepackage[russian,english]{babel}

\usepackage[T2A]{fontenc}
\usepackage[utf8]{inputenc}

\usepackage{setspace}

\usepackage{float}
\usepackage[newfloat]{minted}
\setminted{
    linenos=true,
    autogobble,
}
\usepackage{caption}

\newenvironment{codelisting}{\captionsetup{type=listing}}{}
\SetupFloatingEnvironment{listing}{name=Программный Код}

\usepackage{amsmath}
\usepackage{tikz}
\usepackage{hyperref}
\usepackage{hyphenat}

\usepackage{breqn}

\usepackage{graphicx} % Required for inserting images

\usepackage{pgfplots}

\usepackage{xcolor}
\definecolor{Red}{HTML}{CA4F4B}
\definecolor{Blue}{HTML}{2D70B3}
\definecolor{Orange}{HTML}{FA7E19}

\usepackage{wrapfig}

\hypersetup{
  pdfborder = {0 0 0},
  colorlinks=true,
  linkcolor=black
}


\makeatletter
\renewcommand{\l@section}{\@dottedtocline{1}{0.0em}{0.0em}}
\renewcommand{\l@subsection}{\@dottedtocline{2}{1.4em}{5.5em}}
\renewcommand{\l@subsubsection}{\@dottedtocline{3}{7.4em}{4.5em}}
\makeatother


\renewcommand{\thesection}{}

\renewcommand{\thesubsection}{Задание \arabic{subsection}.}

\addtocontents{toc}{\protect\renewcommand{\protect\thesubsection}{\arabic{subsection}}}


\begin{document}

\begin{titlepage}
    \centering
    \vspace*{0.5 cm}
    \textsc{\LARGE Математический анализ}\\[1.0 cm]
    \textsc{\Large Первый модуль}\\[0.5 cm]
    \textsc{\large 2022-2023 учебный год}\\[1.5 cm]
    \rule{\linewidth}{0.1 mm} \\[0.4 cm]
    { \huge \bfseries Интеграл функции от одной переменной}\\[0.2 cm]
    \rule{\linewidth}{0.1 mm} \\[2 cm]
    \begin{minipage}{0.6\textwidth}
        \begin{flushleft} \large
            \emph{Авторы:}\\
            Бушмелев Константин 1.1,\\ 
            Нигаматуллин Степан 1.1,\\
            Курочка Дмитрий 1.3,\\
            Данилов Тимофей 1.1
        \end{flushleft}
    \end{minipage}~
    \begin{minipage}{0.4\textwidth}
        \begin{flushright} \large
            \emph{Дата:} \\
            27 мая 2023
        \end{flushright}
    \end{minipage}\\[2 cm]
    {\large Университет ИТМО}\\[2 cm]
    \vfill
\end{titlepage}

\renewcommand{\contentsname}{Содержимое}
\tableofcontents
\newpage


\section{Задания}

\subsection{Интегральная сумма}

Исследуйте интегральную сумму функции $f(x) = \frac{1}{\cos^2{x}}$, заданной на отрезке $[-\frac{\pi}{4}, \frac{\pi}{4}]$.

\subsection{Площадь фигуры}

Найдите площадь фигуры, ограниченной лемнискатой Бернулли $p^2 = 8 \cos{2\phi}$.

\subsection{Несобственный интеграл}

Исследуйте несобственный интеграл $\int\limits_{0}^{1}\frac{\arctg{x}}{x^\alpha}dx$ на сходимость при всех значениях параметра $\alpha$.



\subsection{Приложения определенного интеграла}

Определить массу круглого конуса высотой 4 \si{\metre} и диаметром основания 6 \si{\metre}, если плотность конуса в каждой точке равна квадрату расстояния этой точки от плоскости, проходящей через вершину конуса параллельно его основанию.

\subsection{Приближенные вычисления определенного интеграла}

Вычислить заданный определенный интеграл с помощью нескольких указанных численных методов. Оцените погрешность каждого метода. Приветствуется наличие кода в Python, реализующего вычисление интеграла. Сделать сравнительную таблицу использованных методов.

Найти приближенное значение интеграла
$ I_{-1}^{3} = \int\limits_{-1}^{3}\frac{dx}{2+x} $
методами прямоугольников, трапеций, парабол, Уэддля при $h = 1$. Оценить погрешности методов.



\renewcommand{\thesubsection}{\arabic{subsection}.}

\chapter{Модуль 1}

\section{Задание 1. Интегральная сумма}

\import{sections/}{1.1}

\section{Задание 2. Площадь фигуры}

\import{sections/}{1.2}

\section{Задание 3. Несобственный интеграл}

\import{sections/}{1.3}

\section{Задание 4. Приложения определенного интеграла}

\import{sections/}{1.4}

\section{Задание 5. Приложения определенного интеграла}

\import{sections/}{1.5}

\section{Модуль 2}

\subsection{Ряд Тейлора}
\import{sections/}{2.1}

\subsection{Ряд Фурье}
\import{sections/}{2.2}

\end{document}
