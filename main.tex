%\documentclass{beamer}
% \documentclass[paper=smartboard]{powerdot}
\documentclass{article}

\usepackage{siunitx}
\usepackage[russian,english]{babel}

\usepackage[T2A]{fontenc}
\usepackage[utf8]{inputenc}

\usepackage{setspace}

\usepackage{float}
\usepackage[newfloat]{minted}
\setminted{
    linenos=true,
    autogobble,
}
\usepackage{caption}

\newenvironment{codelisting}{\captionsetup{type=listing}}{}
\SetupFloatingEnvironment{listing}{name=Программный Код}

\usepackage{amsmath}
\usepackage{tikz}
\usepackage{hyperref}
\usepackage{hyphenat}

\usepackage{breqn}

\usepackage{graphicx} % Required for inserting images

\usepackage{pgfplots}

\usepackage{xcolor}
\definecolor{Red}{HTML}{CA4F4B}
\definecolor{Blue}{HTML}{2D70B3}
\definecolor{Orange}{HTML}{FA7E19}

\usepackage{wrapfig}

\hypersetup{
  pdfborder = {0 0 0},
  colorlinks=true,
  linkcolor=black
}


\makeatletter
\renewcommand{\l@section}{\@dottedtocline{1}{0.0em}{0.0em}}
\renewcommand{\l@subsection}{\@dottedtocline{2}{1.4em}{5.5em}}
\renewcommand{\l@subsubsection}{\@dottedtocline{3}{7.4em}{4.5em}}
\makeatother


\renewcommand{\thesection}{}

\renewcommand{\thesubsection}{Задание \arabic{subsection}.}

\addtocontents{toc}{\protect\renewcommand{\protect\thesubsection}{\arabic{subsection}}}


\begin{document}

\begin{titlepage}
    \centering
    \vspace*{0.5 cm}
    \textsc{\LARGE Математический анализ}\\[1.0 cm]
    \textsc{\Large Первый модуль}\\[0.5 cm]
    \textsc{\large 2022-2023 учебный год}\\[1.5 cm]
    \rule{\linewidth}{0.1 mm} \\[0.4 cm]
    { \huge \bfseries Интеграл функции от одной переменной}\\[0.2 cm]
    \rule{\linewidth}{0.1 mm} \\[2 cm]
    \begin{minipage}{0.6\textwidth}
        \begin{flushleft} \large
            \emph{Авторы:}\\
            Бушмелев Константин 1.1,\\ 
            Нигаматуллин Степан 1.1,\\
            Курочка Дмитрий 1.3,\\
            Данилов Тимофей 1.1
        \end{flushleft}
    \end{minipage}~
    \begin{minipage}{0.4\textwidth}
        \begin{flushright} \large
            \emph{Дата:} \\
            27 мая 2023
        \end{flushright}
    \end{minipage}\\[2 cm]
    {\large Университет ИТМО}\\[2 cm]
    \vfill
\end{titlepage}

\renewcommand{\contentsname}{Содержимое}
\tableofcontents
\newpage


\section{Задания}

\subsection{Интегральная сумма}

Исследуйте интегральную сумму функции $f(x) = \frac{1}{\cos^2{x}}$, заданной на отрезке $[-\frac{\pi}{4}, \frac{\pi}{4}]$.

\subsection{Площадь фигуры}

Найдите площадь фигуры, ограниченной лемнискатой Бернулли $p^2 = 8 \cos{2\phi}$.

\subsection{Несобственный интеграл}

Исследуйте несобственный интеграл $\int\limits_{0}^{1}\frac{\arctg{x}}{x^\alpha}dx$ на сходимость при всех значениях параметра $\alpha$.



\subsection{Приложения определенного интеграла}

Определить массу круглого конуса высотой 4 \si{\metre} и диаметром основания 6 \si{\metre}, если плотность конуса в каждой точке равна квадрату расстояния этой точки от плоскости, проходящей через вершину конуса параллельно его основанию.

\subsection{Приближенные вычисления определенного интеграла}

Вычислить заданный определенный интеграл с помощью нескольких указанных численных методов. Оцените погрешность каждого метода. Приветствуется наличие кода в Python, реализующего вычисление интеграла. Сделать сравнительную таблицу использованных методов.

Найти приближенное значение интеграла
$ I_{-1}^{3} = \int\limits_{-1}^{3}\frac{dx}{2+x} $
методами прямоугольников, трапеций, парабол, Уэддля при $h = 1$. Оценить погрешности методов.



\renewcommand{\thesubsection}{\arabic{subsection}.}

\chapter{Модуль 1}

\section{Задание 1. Интегральная сумма}

% \begin{wrapfigure}{i}{\textwidth}
\begin{center}
    \begin{tikzpicture}
    \begin{axis}[
        xlabel=$x$,
        ylabel={$y$},
        domain=-3:3,
        samples=200,
        axis lines=middle,
        ymin=-1,
        ymax=3
    ]
    \addplot [thick,color=Red] {1/(cos(deg(x)))^2};
    \addplot [dashed] coordinates {(pi/4,0) (pi/4,3)};
    \addplot [dashed] coordinates {(-pi/4,0) (-pi/4,3)};
    \end{axis}
    \end{tikzpicture}
    
    \caption{График функции $f(x) = \frac{1}{\cos^2{x}}$}
    \label{fig:my_label}
\end{center}
% \end{wrapfigure}

\subsection{Интегральная сумма}

Выполненная работа в Desmos доступна по 
\href{https://www.desmos.com/calculator/h7fwirbtli?lang=ru}{ссылке}

\subsubsection*{Ход работы}

\begin{enumerate}
    \item Построили в Desmos интегральную сумму и график функции.
    Сделали все необходимые ползунки для изменения количества точек в разбиении
    и смещения точек внутри элементарных отрезков.
    \item Исследовали ступенчатую фигуру при количестве ступеней 5, 10 и 20. В каждом положении рассматривали фигуру в левом крайнем, правом крайнем и промежуточном положениях точек внутри элементарных отрезков.
\end{enumerate}

\subsubsection*{Заключение}

На графике легко видеть, что при увеличении количества точек в разбиении, ступенчатая фигура наиболее точно повторяет график.
Так как график симметричный, то крайне левое и крайне правое положение точек внутри элементарных отрезков давали зеркальные фигуры. Можно сказать, что при любом количестве точек в разбиении крайне левое или крайне правое положение показывают наименее точное повторение графика, в отличие от промежуточного, среднего, положения.


\subsection{Последовательность интегральных сумм}

Выполненная работа в Desmos доступна по 
\href{https://www.desmos.com/calculator/3zjcyijmmm?lang=ru}{ссылке}

\subsubsection*{Ход работы}

\begin{enumerate}
    \item Задали в Desmos формулу для интегральной суммы и отоюразили на графике множество её значений при разном $n$ - количестве точек в разбиении.
    \item Рассмотрели значения интегральной суммы при росте $n$ и разном положении точек внутри одного отрезка. Это легко проделать, двигая ползунки в Desmos.
    \item Вычислили интеграл от данной функции аналитически:
    
    \begin{aligned}
    & \int\limits_{-\frac{\pi}{4}}^{\frac{\pi}{4}}\frac{1}{\cos^2{x}}dx = \int\limits_{-\frac{\pi}{4}}^{\frac{\pi}{4}}(1+\tg^2{x})dx = x\bigg|\limits_{-\frac{\pi}{4}}^{\frac{\pi}{4}} + \int\limits_{-\frac{\pi}{4}}^{\frac{\pi}{4}}(\tg^2{x})dx = \left|
    \begin{aligned}
    t = \tg x \\ x = \arctg t \\ dx = \frac{1}{1+t^2}dt \\ t_1 = 1, t_2 = -1 
    \end{aligned}
    \right| = \\ 
    &= \frac{\pi}{2} + \int\limits_{-1}^{1}\frac{t^2}{1+t^2}dt = \frac{\pi}{2} + t\bigg|\limits_{-1}^{1} - \int\limits_{-1}^{1}\frac{1}{1+t^2}dt = \frac{\pi}{2} + 2 - \arctg x\bigg|\limits_{-1}^{1} = \\
    &= \frac{\pi}{2} + 2 - (\frac{\pi}{4} - (-\frac{\pi}{4})) = \frac{\pi}{2} + 2 - \frac{\pi}{2} = 2
    \end{aligned}

    \item Изобразили значение интеграла на графике.
\end{enumerate}

\subsubsection*{Заключение}

На графике можно легко увидеть, что чем больше точек в разбиении мы берём, тем точнее интегральная сумма повторяет значение интеграла от данной функции, вычисленного аналитически. Можно заметить, что вне зависимости от выбора положения точки внутри элементарного отрезка, при больших $n$ интегральная сумма настолько приближается к значению интеграла, что без увеличения масштаба различия не видны. При значении $n=100$ интегральная сумма отличается от интеграла не более чем на $0.0002$ при выборе точки в крайне левом положении. При малых значениях $n$ можно заметить, как именно выбор положения точки внутри элементарного отрезка влияет на интегральную сумму. Так как наша функция чётная, то крайне левое и крайне правое положения совпадают. При малых $n$ и выборе крайних положений точки внутри элементарного отрезка интегральная сумма становится заметно больше значения интеграла. При больших $n$ и большом масштабе видно, что интегральная сумма будет превышать значения интеграла, но на малые величины. Также можно подобрать такой положение, что даже при малых $n$ интегральная сумма будет приблизительно равна интегралу.
\newpage

\section{Задание 2. Площадь фигуры}

% \begin{wrapfigure}{i}{\textwidth}
\begin{center}
    \begin{tikzpicture}
    \begin{axis}[
        xlabel=$x$,
        ylabel={$y$},
        domain=-3:3,
        samples=200,
        axis lines=middle,
        ymin=-3,
        ymax=3
    ]
    \addplot[thick, color=Orange, domain=-2.828:2.828]{sqrt(4*sqrt(1+x^2)-x^2-4)};
    \addplot[thick, color=Orange, domain=-2.828:2.828]{-sqrt(4*sqrt(1+x^2)-x^2-4)};
    \addplot[]{0}; % я добавил эту штуку, чтобы показать небольшие пустые поля (domain=-3:3), но это нехорошо. надо пофиксить
    \end{axis}
    \end{tikzpicture}
    
    \caption{Лемниската Бернулли}
    \label{fig:my_label}
\end{center}
% \end{wrapfigure}

Выполненная работа в Desmos доступна по 
\href{https://www.desmos.com/calculator/23qokppxuf?lang=ru}{ссылке}

\subsubsection*{Ход работы}

\begin{enumerate}
    \item На графике с полярными координатами изобразили лемнискату Бернулли.
    
    \item Формула для нахождения интеграла в полярных координатах:
    $$
    S\ =\ \frac{1}{2}\int\limits_{\phi_{1}}^{\phi_{2}}\rho^{2}\left(\phi\right)d\phi$$
    
    \item Вычислим данный интеграл:
    $$
    S\ =4\cdot\ \frac{1}{2}\int\limits_{0}^{\frac{\pi}{4}}8\cos2\theta\ d\theta\ =\ 8\cdot\left(\sin\left(2\cdot\frac{\pi}{4}\right)-\sin\left(2\cdot0\right)\right)=8
    $$
        Интеграл брался от 0 до $\frac{\pi}{4}$, так как период функции $\cos2\theta$ это $\pi$, следовательно четвертью является $\frac{\pi}{4}$. В свою очередь, четверть отрезка мы берём, так как лемниската симметрична относительно центра координат и оси $0Y$.

    
\end{enumerate}

\subsubsection*{Заключение}

Правдоподобность рассчитанной площади можно оценить, взглянув на график лемнискаты в координатах $X0Y$. Видно, что график ограничен прямоугольником размером 6 на 2. Рассмотрим первую четверть. В ней, на графике отмечены два треугольника, общей площадью равной 1. Для оценки вычтем данную площадь, умноженную на 4, из площади прямоугольника. Мы умножали площадь треугольников на 4, так как лемниската симметрична, относительно начала координат и оси $0Y$, а значит график идентичен во всех четвертях. Итого, получаем оценочную площадь равной 8. Так как сам интеграл оказался так же равен 8, то можем считать, что вычисления точны.

\newpage


\section{Задание 3. Несобственный интеграл}

% \begin{wrapfigure}{i}{\textwidth}
\begin{center}
    \begin{tikzpicture}
    \begin{axis}[
        xlabel=$x$,
        ylabel={$y$},
        domain=-2:2,
        samples=200,
        axis lines=middle,
        ymin=-1,
        ymax=3
    ]
    \addplot [thick,color=Red] {rad(atan(x)/x)};
    \addplot [thick,color=Blue] gnuplot {x**2 * atan(x)};
    \addplot [thick,color={rgb,255:red,0;green,128;blue,0}] {rad(atan(x))/(x^3)};
    \node [above] at (axis cs:1.7,0.65) {\rotatebox{-10}{$\alpha = 1$}};
    \node [left] at (axis cs:-0.7,2.0) {\rotatebox{70}{$\alpha = 3$}};
    \node [left] at (axis cs:0,-0.3) {\rotatebox{20}{$\alpha = -2$}};
    \end{axis}
    \end{tikzpicture}
    
    \caption{ $
    f\left(x\right)=\frac{\arctg\left(x\right)}{x^{\alpha}}$}
    \label{fig:my_label}
\end{center}
% \end{wrapfigure}

Выполненная работа в Desmos доступна по 
\href{https://www.desmos.com/calculator/jjtlvpwjjv?lang=ru}{ссылке}

\subsubsection*{Ход работы}

\begin{enumerate}
    \item Особой точкой подынтегральной функции является точка $x_0 = 0$, так как в ней происходит деление на ноль.
    \item На промежутке от 0 до 1 подынтегральная функция всегда принимает положительные значения, при любых $\alpha$.
    \item При $\alpha = 0$ интеграл легко вычисляется. Рассмотрим его:
    
    \begin{aligned}
    &\int\limits_{0}^{1}\frac{\arctg{x}}{x^0}dx = \int\limits_{0}^{1}\arctg{x}dx = x\arctg x\bigg|\limits_{0}^{1} - \frac{1}{2}\left(\ln\left(1 + x^2\right) \bigg|\limits_{0}^{1} \right) = \hfill \\
    &= 1\cdot \arctg{1} - 0\cdot\arctg{0} - \frac{1}{2} \left(\ln{2} - \ln{0}\right) = \frac{\pi -2\ln{2}}{4}
    \end{align}
    
    Отсюда можно сделать вывод, что интеграл сходится при $\alpha = 0$ и его значения $\approx 0.4388$.

    \item Сформулируем признаки сравнения:
    
    \begin{aligned}
        \text{Пусть функции } f(x) \text{ и } g(x) \text{ определены на промежутке } \\
        (A,B) \text{ и удовлетворяют неравенству } f(x)<=\abs(g(x))
        
    \end{aligned}

    \item Сделаем оценку сверху:
    $$
    \frac{\arctg{x}}{x^{\alpha}} < \frac{\frac{\pi}{2}}{x^{\alpha}}
    $$
    Отсюда видно, что при $\alpha<1$ интеграл от правой функции будет сходится, а значит интеграл от меньшей функции тоже сходится.

    \item Оценка снизу, очевидно, $y = 0$.

\end{enumerate}

\subsubsection*{Заключение}

Тут какие-то словечки...

\newpage

\section{Задание 4. Приложения определенного интеграла}

\begin{center}
    \begin{tikzpicture}
    \begin{axis}[
        xlabel=$x$,
        ylabel={$y$},
        domain=0:7,
        samples=200,
        axis lines=middle,
        ymin=0,
        ymax=5
    ]
    \addplot[thick, color=black] {-abs((4/3)*(x-3))+4};
    \addplot [dashed] coordinates {(3,0) (3,4)};
    \addplot [dashed] coordinates {(7,0) (7,4)};
    \addplot [dashed] coordinates {(3,4) (7,4)};
    \addplot [dashed] coordinates {(4.5,2) (7,2)};
    \addplot [dashed] coordinates {(4.875,1.5) (7,1.5)};
    \node [left] at (axis cs:7,3) {$h$};
    \node [left] at (axis cs:7,1.75) {$dx$};
    \node [above] at (axis cs:4,0) {$2R = 6$};
    \node [left] at (axis cs:3,2) {$H = 4$};
    \end{axis}
    \end{tikzpicture}
    
    \caption{Лемниската Бернулли}
    \label{fig:my_label}
\end{center}
\subsubsection*{Ход работы}

\begin{enumerate}
    \item Выделим на конусе малый участок шириной $dh$, верхняя граница которого находится на расстоянии $r$ от плоскости, проходящей через вершину конуса. 

    \item Масса вычисляется по формуле ($m = pV$), где ($p = h^2$) по условию, где $h$ - расстояние от точки на конусе до плоскости.

    \item Можно считать, что плотность на всем выбранном участке одинакова, так как изменение расстояния r будет незначительным. Поэтому массу элементарного участка можно вычислить по формуле ($dm = h^2dV$)

    \item Вычислим объем элементарного учатска конуса. Ввиду его малости, мы можем считать его цилиндром с высотой $dx$, и радиусом основания = $r$.

    \item Из подобия цилиндров следует : $\frac{r}{R} = \frac{h}{H} \Leftrightarrow r = \frac{Rh}{H}$

    \item Объем элементарного участка равен соответственно:

\begin{dmath}
dV = 
2 \cdot \pi \cdot r^2 \cdot dh \Rightarrow dm = 
h^2 \cdot dV = 
h^2 \cdot 2 \cdot \pi \cdot r^2 \cdot dh = 
h^2 \cdot 2 \cdot \pi \cdot (Rh/H)^2 \cdot dh
\end{dmath}

\begin{dmath}
m = 
\int\limits_{0}^{H}(2 \cdot \pi \cdot R^2 \cdot h^4 / H^2 * dh) = 
\int\limits_{0}^{4}(2 \cdot \pi \cdot 9 \cdot \frac{h^4}{4} * dh) = 
\pi \cdot \frac{9}{2} \cdot \int\limits_{0}^{4}(h^4dh) = 
\pi \cdot \frac{9}{2} \cdot \left(\frac{h^5}{5}\right) \Bigg|_{0}^{4} = 
\pi \cdot 0,9 \cdot 1024 \approx 2895.2917
\end{dmath}

\end{enumerate}

\subsubsection*{Заключение}

С помощью интегралов можно вычислять физические величины, значение которых меняется по какому-либо функциональному закону.

\newpage

\section{Задание 5. Приложения определенного интеграла}

\subsubsection*{Ход работы}

\begin{codelisting}
\begin{minted}{Python}
import numpy
from scipy import integrate


def f(x):
    return 1.0 / (2.0 + x)


def rectangular_method(a, b):
    n = b - a  # количество отрезков (при h = 1)
    h = 1
    I = 0  # интегральная сумма
    x = a + h / 2
    for i in range(n):
        x += h  # левая граница
        I += f(x)  # ((b - a) / n) * f(x(i-1) + h / 2)

    return I


def trapezoidal_method(a, b):
    n = b - a  # количество отрезков (при h = 1)
    h = 1
    I = (f(a) + f(b)) / 2  # интегральная сумма
    for i in range(1, n):
        x1 = a + i  # левая граница
        x2 = x1 + 1  # правая граница
        # S = (f(x1) + f(x2)) / 2 * (x2 - x1) # Площадь трапеции на каждом отрезке
        I += f(x1)

    return I

def find_para(xi1, xi2, xi3):
    a = 0
    b = 0
    c = 0
    x = numpy.array([xi1, xi2, xi3])
    y = numpy.array([f(xi1), f(xi2), f(xi3)])
    z = numpy.polyfit(x, y, 2)
    return z[0], z[1], z[2]


def integrte_para(a, b, c, start, end):
    return ((a * end ** 3) / 3 + (b * end ** 2) / 2 + c * end) - \
        ((a * start ** 3) / 3 + (b * start ** 2) / 2 + c * start)


def simpson_method(a, b):
    I = 0
    h = 1
    n = (b - a) // (2 * h)
    for i in range(n):
        xi1 = a + 2 * h * i
        xi2 = xi1 + h
        xi3 = xi1 + 2 * h
        a_p, b_p, c_p = find_para(xi1, xi2, xi3)
        I += integrte_para(a_p, b_p, c_p, xi1, xi3)
    return I


def weddle(f, a, b, n):
    h = (b - a) / n
    s = 0.0
    x1 = a
    x2 = a + h
    while (x2 <= b):
        dx = (x2 - x1) / 6
        s = s + (h / 20) * (f(x1) + 5 * f(x1 + dx) + f(x1 + 2 * dx) + \
        6 * f(x1 + 3 * dx) + f(x1 + 4 * dx) + 5 * f(x1 + 5 * dx) + f(x1 + 6 * dx))
        x1 = x2
        x2 = x1 + h
    return s

a = -1
b = 3
result = rectangular_method(a, b)
predicted = integrate.quad(f, a, b)[0]

print("{} rectangular_method || fault {}".format(result, predicted - result))

result = trapezoidal_method(a, b)

print("{} trapezoidal_method || fault {}".format(result, predicted - result))

result = integrate.quad(f, a, b)

print("{} library".format(result))

result = simpson_method(a, b)

print("{} simpson_method || fault {}".format(result, predicted - result))

result = weddle(f, a, b, 4)

print("{} weddle || fault {}".format(result, predicted - result))
\end{minted}
\end{codelisting}

\subsubsection*{Вывод программы}

\begin{verbatim}
1.0897546897546897 rectangular_method || fault 0.5196832226794117
1.6833333333333333 trapezoidal_method || fault -0.07389542089923196
(1.6094379124341014, 3.6599536780638574e-09) library
1.6222222222222213 simpson_method || fault -0.012784309788119952
1.6094401258400164 weddle || fault -2.2134059149969687e-06
\end{verbatim}

\subsubsection*{Заключение}

Мы на практике поработали с численными методами вычислени определенных интегралов:

\chapter{Модуль 2}

\subsection{Ряд Тейлора}
Исследуйте ряд Тейлора функции $f(x)$ в точке $x_0$. Изобразите графически несколько различных  частичных сумм ряда и график исходной функции. Проведите анализ полученных результатов.

\subsubsection*{Ход работы}

\begin{enumerate}
    \item В заданной точке разложили  функцию в ряд Тейлора.
	
    $f(x) = (2 - e^{(x-2)})^2 = 4 - 4e^{(x-2)} + e^{2(x-2)}, x_0 = 2$

    Мы умеем раскладывать $g(x) = e^{x}$, если $x\to0$ \Rightarrow $\text{мы умеем раскладывать} g(x) = e^{x-2}\text{, если} x \to 2$.

    Получим:

    $e^{x} =\sum_{n=0}^{\infty} \frac{x^n}{n!} \quad \text{при} \quad x \to 0$

    $e^{x-2} =        \sum_{n=0}^{\infty} \frac{(x-2)^n}{n!} \quad \text{при} \quad x \to 2$

    $e^{2(x-2)} =        \sum_{n=0}^{\infty} \frac{(x-2)^n2^{n}}{n!} \quad \text{при} \quad x \to 2$

    Получим:

    $f(x) = 4 + \sum_{n=0}^{\infty}(\frac{2^{n}}{n!} - \frac{4}{n!})(x-2)^{n} = 4 + \sum_{n=0}^{\infty}\frac{2^{n}-4}{n!}(x-2)^{n}$
    
    \item Нашли область сходимости полученного ряда к функции $f(x)$ - $(-\infty, \infty)$
    \item Построили графики частичных сумм ряда Тейлора (полиномов Тейлора) и график функции.

    Выполненная работа в Desmos доступна по 
\href{https://www.desmos.com/calculator/0imxfkpgnf?lang=ru}{ссылке}

\subsection{Ряд Фурье}
С помощью разложения в ряд Фурье данной функции в интервале $(-\pi, \pi)$ найдите сумму указанного числового ряда. Изобразите графически три различные частичные суммы разложения функции в ряд Фурье, взяв первые несколько слагаемых ряда, а также исходную функцию.

\begin{enumerate}
    \item Метод прямоугольников заключался в покрытии площади под графиков в прямоугольники, площадь которых мы хорошо знаем.
    \item Метод трапеций заключался в покрытии площади под графиком прямоугольными трапециями, крайние точки которых - точки графика.
    \item Метод парабол заключался в апроксимации равномерных частей графика функции параболами, определенный интеграл которых мы легко находим из формулы Ньютона-Лейбница. 
    \item Метод Уэддля в своей сути - метод парабол, но апроксимация идет по 7 степеням многочлена
\end{enumerate}

Как мы можем вдиеть из вывода программы, методы располагаются от менее точного, к более точному
\subsubsection*{Ход работы}

\begin{enumerate}
    \item Представили функцию $f(x) = x\sin{x}$ ее рядом Фурье. Данная функция не является суммой полученного ряда на всей числовой оси.

    \begin{alligned}
        $f(x) = x\sin{x}\quad T = 2\pi  \quad f(x) \approx \frac{a_0}{2} + \sum_{k=0}^{\infty}a_k\cos{kx} + b_k\sin{kx}$

        Найдем коэффициенты Фурье:
        
        $a_0(f) = \frac{1}{\pi}\int_{-\pi}^{\pi} f(x) \, dx = \frac{2}{\pi}\int_{0}^{\pi} x \sin(x) \, dx = \bigg|u = x, dv = \sin{x}dx, du = dx, v = -\cos{x}\bigg| = \frac{2}{\pi}(-x\cos{x} + \int_{0}^{\pi}\cos(x) \, dx) = \frac{-2}{\pi}x\cos{x}\bigg|_0^{\pi} + \sin{x}\bigg|_0^{\pi} = 2$

        $b_k = 0$, так как функция четная.

        $a_k = \frac{2}{\pi}\int_{0}^{\pi} x\sin{x}\cos{kx} \, dx =  \frac{1}{\pi}\int_{0}^{\pi} x\sin{(k+1)x} \, dx  + \frac{1}{\pi}\int_{0}^{\pi} x\sin{(1-k)x} \, dx = \bigg|u = x, dv = \sin{(k+1)x}dx, du = dx, v = \frac{-\cos{(k+1)x}}{k+1}\bigg| = \frac{x}{\pi}\frac{-\cos{(k+1)x}}{k+1}\bigg|_0^{pi}  + \frac{1}{\pi}\int_{0}^{\pi} \frac{\cos{(k+1)x}}{k+1} \, dx + \frac{x}{\pi}\frac{-\cos{(1-k)x}}{1-k}\bigg|_0^{pi}  + \frac{1}{\pi}\int_{0}^{\pi} \frac{\cos{(1-k)x}}{1-k} \, dx = \frac{(-1)^k}{k+1} + \frac{(-1)^k}{1-k} + \frac{\sin{(k+1)x}}{\pi(k+1)^2}\bigg|_0^{\pi} + \frac{\sin{(1-k)x}}{\pi(1-k)^2}\bigg|_0^{\pi} =     \frac{(-1)^k}{k+1} + \frac{(-1)^k}{1-k} = \frac{-2(-1)^k}{k^2-1}$ 

        
        \item Далее разобьем сумму на сумму первого слагаемого и сумму начиная со второго, так как при $n = 1$ было деление на 0 в $a_n$,поэтому нужно отдельно посчитать $a_1$:

        $a_1 = -\frac{1}2}\cos\left(x\right)$

        \item Ряд Фурье: $1\ -\frac{1}{2}\cos\left(x\right)+\sum_{n=2}^{\infty}\left(-\frac{2\left(-1\right)^{n}}{n^{2}-1}\cos\left(nx\right)\right)$

        \item     Выполненная работа в Desmos доступна по 
\href{https://www.desmos.com/calculator/t1pwcmyzo2?lang=ru}{ссылке}

        \item Искомая сумма ряда:
        $\sum_2^{\infty}\frac{(-1)^n}{n^2-1}$. Выразим ее из ряда Фурье:

        $x\sin{x} = 1\ -\frac{1}{2}\cos\left(x\right)+\sum_{n=2}^{\infty}\left(-\frac{2\left(-1\right)^{n}}{n^{2}-1}\cos\left(nx\right)\right)$

        $x\sin{x} - 1 + \frac{1}{2}\cos\left(x\right) = \sum_{n=2}^{\infty}\left(-\frac{2\left(-1\right)^{n}}{n^{2}-1}\cos\left(nx\right)\right)$

        $\frac{-x}{2}\sin{x} + \frac{1}{2} - \frac{\cos{x}}{4} = \sum_{n=2}^{\infty}\left(\frac{\left(-1\right)^{n}}{n^{2}-1}\cos\left(nx\right)\right)$

        В точке $x_0 = 0$:

        $\frac{1}{4} = \sum_{n=2}^{\infty}\left\frac{\left(-1\right)^{n}}{n^{2}-1}\right$
    \end{alligned}


    
\end{enumerate}

\end{document}