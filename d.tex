%\documentclass{beamer}
% \documentclass[paper=smartboard]{powerdot}
\documentclass{article}

\usepackage{siunitx}
\usepackage[russian,english]{babel}

\usepackage[T2A]{fontenc}
\usepackage[utf8]{inputenc}

\usepackage{setspace}

\usepackage{amsmath}

\usepackage{hyperref}
\usepackage{hyphenat}

\usepackage{breqn}

\usepackage{graphicx} % Required for inserting images

\usepackage{pgfplots}

\usepackage{xcolor}
\definecolor{Red}{HTML}{CA4F4B}
\definecolor{Blue}{HTML}{2D70B3}
\definecolor{Orange}{HTML}{FA7E19}

\usepackage{wrapfig}

\hypersetup{
  pdfborder = {0 0 0},
  colorlinks=true,
  linkcolor=black
}


\makeatletter
\renewcommand{\l@section}{\@dottedtocline{1}{0.0em}{0.0em}}
\renewcommand{\l@subsection}{\@dottedtocline{2}{1.4em}{5.5em}}
\renewcommand{\l@subsubsection}{\@dottedtocline{3}{7.4em}{4.5em}}
\makeatother


\renewcommand{\thesection}{}

\renewcommand{\thesubsection}{Задание \arabic{subsection}.}

\addtocontents{toc}{\protect\renewcommand{\protect\thesubsection}{\arabic{subsection}}}


\begin{document}

\begin{titlepage}
    \centering
    \vspace*{0.5 cm}
    \textsc{\LARGE Математический анализ}\\[1.0 cm]
    \textsc{\Large Первый модуль}\\[0.5 cm]
    \textsc{\large 2022-2023 учебный год}\\[1.5 cm]
    \rule{\linewidth}{0.1 mm} \\[0.4 cm]
    { \huge \bfseries Интеграл функции от одной переменной. Вариант 3}\\[0.2 cm]
    \rule{\linewidth}{0.1 mm} \\[2 cm]
    \begin{minipage}{0.6\textwidth}
        \begin{flushleft} \large
            \emph{Автор:}\\
            Курочка Дмитрий 1.3
        \end{flushleft}
    \end{minipage}~
    \begin{minipage}{0.4\textwidth}
        \begin{flushright} \large
            \emph{Дата:} \\
            3 июня 2023
        \end{flushright}
    \end{minipage}\\[2 cm]
    {\large Университет ИТМО}\\[2 cm]
    \vfill
\end{titlepage}

\renewcommand{\contentsname}{Содержимое}
\tableofcontents
\newpage


\section{Задания}

\subsection{Значение функции}

Вычислить приближенное значение функции $f(x) = e^x$ в точке $x_0 = -\frac{1}{5}$ c точностью 0,0001.

\subsection{Вычислить интеграл}

Разлагая подынтегральную функцию в степенной ряд вычислить приблеженно интеграл $\int\limits_{0}^{0.5} \sin{x^2}dx$ с точностью 0,0001.

\subsection{Степенной ряд}

Найти в виде степенного ряда решение дифференциального уравнения, удовлетворяющего заданным начальным условиям. Ограничиться четырьмя членами ряда. 
\begin{center}
    $y' = \ln{y} + 1 + x; y(0) = 1$
\end{center}

\renewcommand{\thesubsection}{\arabic{subsection}.}


\section{Задание 1. Значение функции $f(x) = e^x$}

\begin{enumerate}
    \item $f\left(x\right)\ =\sum\limits_{n=0}^{\infty}\frac{x^n}{n!}$
    \item $e^{-\frac{1}{5}} = \sum\limits_{n=0}^{\infty}\frac{(-\frac{1}{5})^n}{n!} = \sum\limits_{n=0}^{\infty}\frac{(-1)^n}{5^nn!}$
    \item Заметим, что данный ряд сходится на всей числовой оси, следовательно, в частности, его можно применять если $x \Rightarrow -\frac{1}{5}$, Заметим так же, что точность слагаемого ряда с 4 степенью составляет $\frac{1}{15000}$, следовательно, при нашей точности, брать слагаемые выше 4 степени не имеет смысла. Также важно отметить, что ряд имеет лейбницовский тип, поэтому мы можем отбросить остаточный многочлен.
    \item Приближенное значение: $e^{-\frac{1}{5}} \approx  \sum\limits_{n=0}^{5}\frac{(-1)^n}{5^nn!} = 0.8187$
\end{enumerate}

% \subsubsection*{Заключение}

% Тут какие-то словечки...


\section{Задание 2. Вычислить интеграл $\int\limits_{0}^{0.5} \sin{x^2}dx$}

Разложим для начала функцию

\[
\begin{split}
    &g(x) = \sin{x}
     = \sum_{n=0}^{k}\frac{\left(-1\right)^{n}x^{2n+1}}{\left(2n\ +\ 1\right)!}\\
    &g(x^2) = \sin{x^2}
     = \sum_{n=0}^{k}\frac{\left(-1\right)^{n}x^{4n+2}}{\left(2n\ +\ 1\right)!} \\
    &\int\limits_{0}^{0.5}\sin{x^2}dx = \int\limits_{0}^{0.5}\sum_{n=0}^{k}\frac{(-1)^{n}x^{4n+2}}{(2n+1)!}dx
\end{split}
\]

Воспользуемся свойствами определенного интеграла, распишем сумму, так же воспользуемся формулой Ньютона-Лейбница, получим:

\[
\begin{split}
&\int\limits_{0}^{0.5}\sin{x^2}dx = \sum_{n=0}^{k}\int\limits_{0}^{0.5}\frac{(-1)^{n}x^{4n+2}}{(2n+1)!}dx = \sum_{n=0}^{k}\frac{(-1)^{n}x^{4n+3}}{(4n+3)(2n+1)!}\bigg|\limits_{0}^{0.5} = \\
&=\sum_{n=0}^{k}\frac{(-1)^n}{(2n+1)!}\frac{1}{(4n + 3)2^{4n+3}} = \frac{1}{3*2^3} - \frac{1}{6 * 7 * 2^7} + \frac{1}{120 * 11 * 2^{11}} + r_n 
\end{split}
\]

Итого: 
\\
$\int\limits_{0}^{0.5}\sin{x^2}dx \approx 0.04148 \approx 0.0415$

\section{Задание 3. Степенной ряд}

$\quad\sqsupset y = y(0) + \frac{y'(0)x}{1!} + \frac{y''(0)x^2}{2!} + \frac{y'''(0)x^3}{3!}$ - Ряд Тейлора для данной функции в точке $x_0 = 0$, ограничивающийся только 4 первыми слагаемыми, итак значения 0 производной мы знаем, рассамотрим производные далее:
\begin{align}
    y'' = (y')' = (\ln{y})' + x' = \frac{y'}{y} + 1\\
    y''' = (y'')' = \frac{y''y - (y'^2)}{y^2}
\end{align}
\begin{align}    
    y(0) = 1
    y'(0) = 1
    y''(0) = 2
    y'''(0) = 1
\end{align}
Итого получим:
$y = 1 + x + x^2 + \frac{x^3}{6}$

\end{document}
